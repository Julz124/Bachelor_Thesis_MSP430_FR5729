% !TEX encoding = UTF-8 Unicode
% !TEX root =  Bachelorarbeit.tex

% Zeilenabstand 1,5 Zeilen ---------------------------------------------------------------------
\onehalfspacing{}


% Seitenränder ------------------------------------------------------------------------------------
\setlength{\topskip}{\ht\strutbox} % behebt Warnung von geometry
\geometry{paper=a4paper,left=30mm,right=30mm,top=30mm}


% Kopf- und Fußzeilen --------------------------------------------------------------------------
\pagestyle{scrheadings}

% Kopf- und Fußzeile auch auf Kapitelanfangsseiten
\renewcommand*{\chapterpagestyle}{scrheadings} 

% Schriftform der Kopfzeile
\renewcommand{\headfont}{\normalfont}


% Kopfzeile für einseitiges Dokument:
\ihead{%\small{\textsc{\titel}	\\
	%\untertitel
	%\\[2ex]
	\textit{\headmark}%}
}
\chead{}
\rohead{\includegraphics[scale=0.3]{\logo}}
\setlength{\headheight}{21mm} % Höhe der Kopfzeile
% Kopfzeile über den Text hinaus verbreitern
\setheadwidth[0pt]{textwithmarginpar} 
\setheadsepline[text]{0.4pt} % Trennlinie unter Kopfzeile


%% Kopfzeile für Zweiseitiges Dokument:
%\ihead{\textit{\leftmark}}
%\chead{}
%\ohead{}
%\lehead{\includegraphics[scale=0.25]{\logo}}
%\rohead{\includegraphics[scale=0.25]{\logo}}
%\setheadsepline[text]{0.4pt} % Trennlinie unter Kopfzeile


% Fußzeile
%\ifoot{\copyright\ \autor}
\cfoot{}
\ofoot{\pagemark \\[4ex]}


% sonstige typographische Einstellungen ---------------------------------------------------
% erzeugt ein wenig mehr Platz hinter einem Punkt
\frenchspacing{}

% Schusterjungen und Hurenkinder vermeiden
\clubpenalty = 10000
\widowpenalty = 10000 
\displaywidowpenalty = 10000

% Quellcode-Ausgabe formatieren
\lstset{numbers=left, numberstyle=\tiny, numbersep=5pt, breaklines=true}
\lstset{emph={square}, emphstyle=\color{red}, emph={[2]root,base}, emphstyle={[2]\color{blue}}}

% Fußnoten fortlaufend durchnummerieren
\counterwithout{footnote}{chapter}

