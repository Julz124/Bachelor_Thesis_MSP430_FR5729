% !TEX encoding = UTF-8 Unicode
% !TEX root =  ../Bachelorarbeit.tex

\chapter{Einleitung}
\label{cha:Einleitung}


\section{Das Ziel dieser Arbeit}
\label{sec:ZielDerArbeit}

Diese Bachelor-Thesis befasst sich mit der Entwicklung einer \Fachbegriff{Ein Mechanismus zur ereignisorientierten Unterbrechung des normalen Programmablaufs}{interruptgesteuerten}[interruptgesteuert] Benutzerschnittstelle auf einem \Fachbegriff{Ein Mikrocontroller, der f\"ur energieeffiziente Anwendungen optimiert ist. Typischerweise eingesetzt in batteriebetriebenen Embedded-Systemen}{Low-Power-Mikrocontroller}, zur \"Uberwachung und Manipulation von \Fachbegriff{Speicherzellen, die fl\"uchtig sind und ihre Inhalte beim Ausschalten verlieren}{Registern}[Register] und Speicherzellen in \Abkuerzung{Random Access Memory}{RAM} und \Abkuerzung{Ferroelectric Random Access Memory}{FRAM}.

Im Zuge des Arbeitsauftrags wird ein unabh\"angiges \Fachbegriff{Eine funktionale Einheit innerhalb eines gr\"o{\ss}eren Systems, die separat entwickelt und gewartet werden kann}{Modul} entwickelt, welches nach Wunsch aktiviert oder deaktiviert wird.


\section{Die Umgebung, in der die Arbeit entstand}
\label{sec:EntstehungsUmgebungArbeit}

Die Entwicklung der Software geschah in Absprache mit Herr Prof. Dr. Iren\"aus Schoppa, welcher ein zus\"atzliches Tool f\"ur Studenten im Praktikum von Microprozessorsysteme ben\"otigt.

Als Entwicklungsbasis kam der in \Abkuerzung{Microprozessorsysteme}{MIPS} herangezogene MSP430FR5729 von Texas Instruments zum Einsatz, welcher bereits ein ausgereifter und etablierter \Abkuerzung{Low-Power-Mikrocontroller}{LP-MCU} ist. Viele Technologien, die diesem Prozessor zugrunde liegen, werden in dieser Arbeit wissentlich nicht behandelt, um den Rahmen nicht zu sprengen und sich auf das wesentliche zu konzentrieren.


\section{Der Aufbau dieser Arbeit}
\label{sec:AufbauDieserArbeit}

\begin{description}

	\item[Grundlagen, Technologien und Evaluation:] Im \Kapitel{UeberblickEntwicklungsplattform} werden die zur Planung und Umsetzung verwendeten Grundlagen und Technologien erarbeitet. Die grundlegenden Eigenschaften und der Aufbau des Softwareentwicklungsprozesses auf einem LP-MCU werden erkl\"art und die ben\"otigten Dokumentationen referenziert.
	
	\item[Konzeptionierung der Benutzerschnittstelle:] Dieses Kapitel umfasst die Dokumentation der gesamten Planungsphase des Observer-Moduls. Hier wird eine \"Ubersicht \"uber die L\"osungsfindung geschaffen und anschlie{\ss}end die zur Planung erforderlichen Dokumente und Diagramme angefertigt.
	
	\item[Die Entwicklung des Observers:] Dieses Kapitel enth\"alt die Dokumentation der tats\"achlichen Programmierung des Software-Moduls. Hier wird die Implementation gekl\"art und der Verlauf der Entwicklung anhand von Beispielen schrittweise abgearbeitet.
	
	\item[Fazit und kritische Bewertung:] Im Fazit werden die gemachten Erfahrungen und die Ergebnisse der Planung und Entwicklung abschlie{\ss}end zusammengefasst und kritisch bewertet. Zus\"atzlich wird ein kleiner Ausblick auf Erweiterungsm\"oglichkeiten und m\"ogliche Optimierungsschritte unternommen.

\end{description}


\section{Viele Informationen, wenige Quellen \dots}
\label{sec:Quellenlage}

Grunds\"atzlich war es einfach geeignete Quellen zu den Themen rund um die Technologien zu finden, da -- wie bereits erw\"ahnt -- die Entwicklungsplattform und der Microcontroller weitesgehend etabliert sind. Allerdings k\"onnen alle wichtigen Informationen aus erster Hand, von dem Hersteller entnommen werden, weshalb andere Quellen unn\"otig erscheinen.