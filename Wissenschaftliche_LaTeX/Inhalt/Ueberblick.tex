% !TEX encoding = UTF-8 Unicode
% !TEX root =  ../Bachelorarbeit.tex

\chapter{\"Uberblick}
\label{UeberblickEntwicklungsplattform}

Die Auswahl einer geeigneten Entwicklungsplattform bildet die Grundlage f\"ur die erfolgreiche Implementierung und Evaluierung eingebetteter Systeme. Im Rahmen dieser Arbeit dient der MSP430FR5729 von Texas Instruments als zentrale Hardwarekomponente. Dessen Architektur und Funktionalit\"aten werden in den folgenden Abschnitten n\"aher betrachtet.

Der MSP430FR5729 ist ein Low-Power-Microcontroller  \Fachbegriff[Mikrocontroller mit 16-Bit-Registerbreite und reduzierter Befehlssatzarchitektur (Reduced Instruction Set Computer)]{16-Bit-RISC-Microcontroller} von Texas Instruments mit einer Maximalen Taktfrequenz von Acht \Fachbegriff[Ma{\ss}einheit f\"ur die Frequenz und entspricht einer Million Schwingungen pro Sekunde (1 MHz = 10$^{6}$ Rechenschritte).]{Megaherz}. Eingebaute Low-Power-Modi (\Abkuerzung{Low Power Mod}{LPM}s), (Auflistung aller Modi in \Abbildung{operation_modes}) erm\"oglichen \ua niedrigere Taktfrequenzen und das deaktivieren von Oszillatoren, wodurch er sich besonders gut f\"ur energieeffiziente Anwendungen im Bereich eingebetteter Systeme eignet. \Zitat[S. 43, Kap. 6.3, S. 35, Kap. 1.4 \& S. 37, Kp. 1.4.1]{ti:slase35c, ti:slau272d}

\begin{figure}[h!]
	\centering
	\includegraphics[width=1.0\textwidth]{../Bilder/Operating_Modes.png}
	\caption{Operating Modes\\\Zitat[S. 37, Kap. 1.4, Tab. 1-2]{ti:slau272d}}
	\label{fig:operation_modes}
\end{figure}

Der Mikroprozessor besitzt 16 Kilobyte an nicht-fl\"uchtigen FRAM, sowie ein Kilobyte \Fachbegriff[Schnellster, fl\"uchtiger Speicher mit geringer Kapazit\"at, bestehend aus Flip-Flops welcher meist direkt in der CPU mit eingebaut ist.]{statischen Arbeitsspeicher} (\Abkuerzung{Static Random Access Memory}{SRAM}). 

Die Versorgungsspannung betr\"agt 2 bis 3,6 Volt wobei ebenfalls verschiedene Low-Power-Modi verwendet werden k\"onnen, um den Stromverbrauch zunehmend zu minimieren. Diese beeinflussen den sp\"ateren Umgang mit Timer-Interrupts, weil sie den Energieverbrauch im Wartezustand beeinflussen. \Zitat[S. 26, Kap. 5.20]{ti:slase35c}

Des Weiteren besitzt der Chip F\"unf Interne 16-Bit Timer mit jeweils Sieben\\\NeuerBegriff{Capture and Compare} Registerbl\"ocken. Diese internen Timer stellen eine zentrale Komponente f\"ur die Realisierung pr\"aziser Zeitgesteuerter Funktionen und die Generierung von Interrupts dar, welche im nachfolgenden Kapitel \ref{TIMER&ISR} tiefgreifender erl\"autert werden.

Zur externen Kommunikation sind Protokolle wie \Abkuerzung{Universal Asynchronous Receiver Transmitter}{UART}, \Abkuerzung{Inter-Integrated Circuit}{I$^{2}$C} und \Abkuerzung{Serial Peripheral Interface}{SPI} integriert, welche mit 32 Programmierbaren \Abkuerzung{General Purpose Input/Output}{GPIO}-Pins angeschlossen werden k\"onnen. Kommunikationsschnittstellen sind f\"ur die Interaktion mit der Au{\ss}enwelt und Peripherieger\"aten von hoher Bedeutung. Eine detailliertere Ausarbeitung des \Fachbegriff[Serielle Schnittstelle in Mikrocontrollern von Texas Instruments, die verschiedene Kommunikationsprotokolle (\zB UART, SPI, I$^{2}$C) unterst\"utzt.]{enhanced Universal Serial Communication Interface} (\Abkuerzung{enhanced Universal Serial Communication Interface}{eUSCI}) in Kapitel \ref{eUSCI}. \Zitat[S. 1, Kap. 1.1]{ti:slase35c}

\begin{figure}[h!]
	\centering
	\includegraphics[width=1.0\textwidth]{../Bilder/FunctionalBlockDiagram_MSP430FR5729.png}
	\caption{Block Diagramm MSP430FR5729\\Mikrocontroller \Zitat[S. 2, Kap. 1.4]{ti:slase35c}}
	\label{fig:BlockDiagramm_msp430}
\end{figure}

\newpage
\Abbildung{BlockDiagramm_msp430} zeigt ein vollst\"andiges Block Diagramm des Mikroprozessors, welches noch einige weitere Eigenschaften, Funktionen und Subsysteme auflistet. \AI


\section{Timer und Interrupt Service Routinen (ISR)}
\label{TIMER&ISR}

Timer und Interrupt Service Routinen (\Abkuerzung{Interrupt Service Routine}{ISR}s) stellen einen fundamentalen Baustein moderner eingebetteter Systeme dar. Sie erm\"oglichen pr\"azise, zeitgesteuerte Funktionen als auch das reagieren auf externe Ereignisse. Womit die Realisierungen komplexer, Echtzeitsysteme m\"oglich wird. Im Folgenden wird die Timer-Architektur des MSP430FR5729 und die zugeh\"origen ISR-Mechanismen detailliert betrachtet.

Der MSP430FR5729 verf\"ugt \"uber insgesamt f\"unf 16-Bit-Timer, wobei zwei dem Typ A und drei dem Typ B angeh\"oren. Beide Typen erm\"oglichen vielseitige Zeitsteuerungsfunktionen, weisen jedoch spezifische Unterschiede in ihren Konfigurationsm\"oglichkeiten auf.

Beide Timer-Typen verf\"ugen \"uber einen gemeinsamen 16-Bit-Z\"ahler sowie sieben Capture/Compare-Register. Diese Register erm\"oglichen die Implementierung verschiedenster Funktionen. Die Capture-Funktionalit\"at dient dazu, den aktuellen Z\"ahlerwert bei einem externen oder internen Ereignis pr\"azise zu erfassen. Dies ist beispielsweise n\"utzlich f\"ur die Messung von Pulsweiten oder Frequenzen. Die Compare-Funktionalit\"at hingegen erlaubt den Vergleich des aktuellen Z\"ahlerstandes mit einem in den Compare-Registern hinterlegten Wert. Bei einer \"ubereinstimmung kann eine konfigurierbare Aktion ausgel\"ost werden, wie beispielsweise das Setzen oder R\"ucksetzen eines Ausgangspins oder das Generieren eines Interrupts. Die vielseitigen Einstellungsm\"oglichkeiten dieser Register erlauben die Realisierung komplexer Zeitgesteuerter Aufgaben. \Zitat[S. 333, Kap. 11 \& S. 355, Kap. 12, S. 287, Kap. 8.3 \& S. 194, Kap. 6.8.2]{ti:slau272d,davies:msp430}

Die Timer des Typs B weisen im Vergleich zu dem Timer des Typs A, erweiterte Konfigurationsm\"oglichkeiten auf. Darunter f\"allt die Konfigurierbarkeit der Timer-L\"ange auf 8, 10, 12 oder 16 Bit, was eine flexible Anpassung der Z\"ahlaufl\"osung und der \"uberlaufperiode f\"ur unterschiedliche Aufl\"osungen erm\"oglicht. Weiterhin sind alle Capture/Compare-Bl\"ocke doppelt gepuffert. Diese doppelte Pufferung erlaubt das Laden neuer Vergleichswerte, w\"ahrend eines aktiven Z\"ahlzyklus, wodurch unerw\"unschte Effekte oder Inkonsistenzen in den Ausgangssignalen vermieden werden. Zudem k\"onnen alle Ausg\"ange auf einen hochohmigen Zustand umgeschaltet werden, was in bestimmten Applikationen vorteilhaft sein kann. Ein weiterer wichtiger Unterschied besteht darin, dass die Capture/Compare-Eing\"ange nicht synchronisiert sind und somit asynchron zu dem internen Takt des Timers operieren k\"onnen, was in bestimmten Szenarien die Erfassung externer Ereignisse erleichtert. \Zitat[S. 356, Kap. 12.1.1, S. 353, Kap. 8.9]{ti:slau272d, davies:msp430}

F\"ur die pr\"azise Steuerung und Ereignisbehandlung bieten die Timer verschiedene Betriebsarten, die im Folgenden n\"aher erl\"autert werden.

\newpage
\subsection{Timer Z\"ahlweisen}
\label{Timer_CountMode}

Der Z\"ahlmodus, bestimmt die interne Z\"ahlweise des Timers. Die Timer unterst\"utzen typischerweise mehrere Varianten dieses Modus, um unterschiedlichen Anforderungen gerecht zu werden. \Zitat[S. 291, Kap. 8.3.1]{davies:msp430}

\begin{itemize}
	\item \textbf{Up Mode:} Im Up Mode (Additive Z\"ahlweise) beginnt der Z\"ahler bei Null und inkrementiert seinen Wert mit jedem Taktimpuls der gew\"ahlten \Fachbegriff[Eine Referenz auf ein periodisches Zeitsignal um zeitliche Abl\"aufe zu synchronisieren; typischerweise in Form von Quarzoszillatoren oder externen Taktsignalen.]{Clock-Source}. Er erreicht einen vordefinierten Maximalwert, der im Compare-Register gespeichert ist, und beginnt dann wieder von Null zu z\"ahlen. Ein \"uberlauf-Interrupt wird generiert, sobald der Z\"ahler den Wert von CCR0 erreicht. Dieser Modus eignet sich ideal f\"ur die Erzeugung periodischer Ereignisse oder die Messung von Zeitintervallen bis zu einem bestimmten Grenzwert. Beispielsweise kann durch die Wahl einer geeigneten Clock-Source und eines passenden Wertes im Compare-Register eine pr\"azise Zeitbasis f\"ur periodische Aufgaben geschaffen werden. \Zitat[S. 337, Kap. 11.2.3.1 \& S. 359, Kap. 12.2.3.1, S. 330, Kap. 8.6]{ti:slau272d, davies:msp430}
	
	\item \textbf{Continuous Mode:} Der Continuous Mode l\"asst den Z\"ahler von Null bis zum maximal m\"oglichen Wert (FFFFh f\"ur 16-Bit-Timer) z\"ahlen und anschlie{\ss}end wieder bei Null beginnen. Ein \"uberlauf-Interrupt wird generiert, wenn der Z\"ahler vom Wert von FFFFh auf 0 \"uberl\"auft. \Zitat[S. 338, Kap. 11.2.3.2 \& S. 360, Kap. 12.2.3.2]{ti:slau272d} Dieser Modus ist besonders n\"utzlich, wenn l\"angere, voneinander unabh\"angige Zeitintervalle zu messen oder wenn eine freilaufende Zeitbasis ben\"otigt wird, um Ereignisse in Bezug auf den Z\"ahlerstand, ohne einen periodischen Neustart durch das Compare-Register, zu erfassen. \Zitat[S. 338, Kap. 11.2.3.3 \& S. 360, Kap. 12.2.3.3, S. 318, Kap. 8.5]{ti:slau272d, davies:msp430}

	\item \textbf{Up/Down Mode:} Der Up/Down Mode (Auf-/Abw\"artsz\"ahlmodus) kombiniert das Auf- und Abz\"ahlen. Der Z\"ahler beginnt bei Null, z\"ahlt Zyklisch bis zum festgelegten Wert im Compare-Register und dann wieder bis Null herunter. Ein \"uberlauf-Interrupt wird generiert, wenn der Z\"ahler den Wert von CCR0 erreicht, und ein weiterer Interrupt (sofern aktiviert) kann beim Erreichen von Null gesetzt werden. \Zitat[S. 339, Kap. 11.2.3.4 \& S. 361, Kap. 12.2.3.4]{ti:slau272d} Dieser Modus erzeugt eine symmetrische \Fachbegriff[Ein Verfahren zur Steuerung der Leistungszufuhr, bei dem die mittlere Ausgangsleistung durch Variieren des Abtastverh\"altnisses eines Rechtecksignals reguliert wird.]{Pulsweitenmodulation} (\Abkuerzung{Pulsweitenmodulation}{PWM}) und wird h\"aufig in Anwendungen zur Motorsteuerung oder zur Erzeugung pr\"aziser analoger Ausgangssignale eingesetzt. \Zitat[S. 340, Kap. 11.2.3.5 \& S. 362, Kap. 12.2.3.5, S. 349, Kap. 8.7]{ti:slau272d, davies:msp430}
\end{itemize}

Die Wahl eines geeigneten Modus h\"angt stark von der spezifischen Anwendung ab. F\"ur einfache Zeitmessungen oder periodische Aufgaben ist der Up Mode oft ausreichend, w\"ahrend der Continuous Mode f\"ur l\"angere Intervalle oder als Basis f\"ur komplexere Zeitsteuerungen dient. Der Up/Down Mode hingegen findet seine Anwendung prim\"ar in der Erzeugung von Steuersignalen.

\subsection{Capture-Mode}
\label{Timer_CaptureMode}

Der Capture Mode erm\"oglicht es, den aktuellen Wert des Z\"ahlers pr\"azise zu erfassen, wenn ein bestimmtes Ereignis an einem zugeh\"origen Eingangspin auftritt. Der erfasste Z\"ahlerwert wird in einem der Capture-Register (CCR0 bis CCR6) gespeichert. Dies ist besonders n\"utzlich f\"ur die Messung von externen Signalen wie Pulsweiten, Frequenzen oder der Zeit zwischen zwei Ereignissen. Beispiele hierzu in \Abbildung{CaptureModeBeispiele}.

\begin{figure}[h!]
	\centering
	\includegraphics[width=1.0\textwidth]{../Bilder/CaptureMode_Beispiele.png}
	\caption{Capture Mode Einsatzbeispiele\\\Zitat[S. 301, Abb. 8.7]{davies:msp430}}
	\label{fig:CaptureModeBeispiele}
\end{figure}

Die Timer des MSP430FR5729 unterst\"utzen verschiedene Capture-Modi. Diese legen fest, bei welcher Art von Signal\"anderung die Erfassung des Z\"ahlerwertes erfolgt:

\begin{itemize}
	\item \textbf{Capture on rising edge:} Sobald am zugeh\"origen Eingangspin eine steigende Flanke detektiert wird (\"ubergang von Low nach High) wird in diesem Modus der aktuelle Z\"ahlerwert in das Capture-Register geschrieben.

	\item \textbf{Capture on falling edge:} Hier erfolgt die Erfassung des Z\"ahlerwertes am Eingangspin bei einer fallenden Flanke (\"ubergang von High nach Low).

	\item \textbf{Capture on both edges:} Dieser Modus erm\"oglicht die Erfassung des Z\"ahlerwertes sowohl bei steigender als auch fallender Flanken. Dies ist besonders praktisch f\"ur die Messung von Signalperioden oder bei Relevanz beider Flanken eines Signals.
\end{itemize}

Sofern ein Interrupt im entsprechenden Capture-Register aktiviert ist, kann dieser auch Interrupts ausl\"osen. In der zugeh\"origen ISR kann der erfasste Z\"ahlerwert aus dem Capture-Register gelesen und weiterverarbeitet werden. Mehrere Capture-Register innerhalb eines Timers erm\"oglichen die Erfassung und Auswertung mehrerer aufeinanderfolgender Ereignisse, ohne dass der vorherige Wert \"uberschrieben wird. 

Die Konfiguration des Capture Mode umfasst die Auswahl des ausl\"osenden Ereignisses (Flanke) sowie \ggf die Aktivierung des Capture-Interrupts. Die erfassten Zeitstempel im Capture-Register erlauben pr\"azise Messungen und die Analyse externer Signale in eingebetteten Systemen. \Zitat[S. 340, Kap. 11.2.4.1 \& S. 362, Kap. 12.2.4.1, S. 300, Kap. 8.4]{ti:slau272d, davies:msp430}

\subsection{Compare-Mode}
\label{Timer_CompareMode}

Der Compare Mode erm\"oglicht es, den aktuellen Wert des Z\"ahlers kontinuierlich mit den in den Compare-Registern CCR0 bis CCR7 hinterlegten Werten zu vergleichen. Wenn der Z\"ahlerstand mit dem Vergleichswert \"ubereinstimmt, kann \zB ein Interrupt ausgel\"ost oder ein Ausgangspin beeinflusst werden.

Die Compare-Modi bieten verschiedene M\"oglichkeiten, wie der Ausgangspin bei einer \"ubereinstimmung beeinflusst werden soll:

\begin{itemize}
	\item \textbf{Set output on compare:} Bei einer \"ubereinstimmung des Z\"ahlerstandes mit dem Compare-Registerwert wird der zugeh\"orige Ausgangspin auf High gesetzt.

	\item \textbf{Reset output on compare:} Hier wird der Ausgangspin bei \"ubereinstimmung auf Low gesetzt.

	\item \textbf{Toggle output on compare:} In diesem Modus \"andert der Ausgangspin bei jeder \"ubereinstimmung seinen Zustand (von High nach Low oder von Low nach High).

	\item \textbf{Output High:} Der Ausgangspin wird permanent auf High gehalten.

	\item \textbf{Output Low:} Der Ausgangspin wird permanent auf Low gehalten.

	\item \textbf{Set/Reset:} In Kombination mit dem Compare-Register 0 (CCR0) kann ein PWM-Signal erzeugt werden. Beispielsweise kann der Ausgang bei Erreichen des CCR0-Wertes gesetzt und bei Erreichen des CCRn-Wertes zur\"uckgesetzt werden (oder umgekehrt), wobei CCRn die Pulsweite bestimmt.
\end{itemize}

\Abbildung{OutputUnit_UpDown_Mode} zeigt eine m\"ogliche Konfiguration im Z\"ahlmodus Up/Down mit zwei Compare-Registern (TAxCCR1 \& TAxCCR2), eingestellt auf Toggle/Set und Toggle/Reset.

\begin{figure}[h!]
	\centering
	\includegraphics[width=1.0\textwidth]{../Bilder/UpDown_ModeBsp.png}
	\caption{Ausgabeeinheit im Up/Down-Modus\\\Zitat[S. 340, Abb. 11-9]{ti:slau272d}}
	\label{fig:OutputUnit_UpDown_Mode}
\end{figure}

\"Ahnlich wie beim Capture Mode erm\"oglicht ein Interrupt der CPU, auf pr\"azise Zeitpunkte zu reagieren und entsprechende Aktionen auszuf\"uhren. Der Compare Mode ist somit ein vielseitiges Werkzeug zur Erzeugung von Steuersignalen, zur Implementierung von Zeitverz\"ogerungen oder zur Synchronisation interner Operationen mit einer pr\"azisen Zeitbasis. \Zitat[S. 342, Kap. 11.2.4.2 \& S. 364, Kap. 12.2.4.2, S. 352, Kap. 8.8]{ti:slau272d, davies:msp430}

Nachdem die verschiedenen Betriebsarten des Timers betrachtet wurden, ist es wichtig zu verstehen, wie die zugeh\"origen Register konfiguriert werden, um die gew\"unschte Funktionalit\"at zu erzielen.


\subsection{Einstellungen der Capture and Compare Register}
\label{CC_Register}

Die Funktionalit\"at der Capture- und Compare-Einheiten wird ma{\ss}geblich durch die Konfiguration ihrer zugeh\"origen Register bestimmt. Hierzu geh\"oren die Aktivierung und Deaktivierung von Interrupts, die Auswahl des Ausgangsmodus (nur f\"ur Compare) sowie die Festlegung des ausl\"osenden Ereignisses.

F\"ur jedes Capture/Compare-Register kann individuell festgelegt werden, ob ein Interrupt ausgel\"ost werden soll, wenn ein entsprechendes Ereignis eintritt. Dies geschieht \"uber spezifische \NeuerBegriff{Interrupt-Enable-Bits} im jeweiligen Capture/Compare-Control-Register (TAxCCTLn oder TBxCCTLn). Durch das Setzen des CCIE-Bits auf Eins oder Null, kann die Generierung eines Interrupts bei einem Capture- oder Compare-Ereignis aktiviert \bzw deaktiviert werden. \Tabelle{tb_ccc_register} fasst alle weiteren Register des Timers B mit ihren Beschreibungen zusammen.


\begin{table}[h!]
	\small
	\centering
	\begin{tabular}{|c|l|c|c|p{8cm}|}
		\hline
		\textbf{Bit} & \textbf{Field} & \textbf{Type} & \textbf{Reset} & \textbf{Description} \\ \hline
		15-14 & CM & RW & 0h & Capture mode \\ \hline
		13-12 & CCIS & RW & 0h & Capture/compare input select. These bits select the TBxCCRn input signal. \\ \hline
		11 & SCS & RW & 0h & Synchronize capture source. This bit is used to synchronize the capture input signal with the timer clock. \\ \hline
		10-9 & CLLD & RW & 0h & Compare latch load. These bits select the compare latch load event.  \\ \hline
		8 & CAP & RW & 0h & Capture-/Compare mode \\ \hline
		7-5 & OUTMOD & RW & 0h & Output mode. \\ \hline
		4 & CCIE & RW & 0h & Capture/compare interrupt enable. This bit enables the interrupt request of the corresponding CCIFG flag. \\ \hline
		3 & CCI & R & Undef & Capture/compare input. The selected input signal can be read by this bit. \\ \hline
		2 & OUT & RW & 0h & Output. For output mode 0, this bit directly controls the state of the output. \\ \hline
		1 & COV & RW & 0h & Capture overflow. This bit indicates a capture overflow occurred. COV must be reset with software.  \\ \hline
		0 & CCIFG & RW & 0h & Capture/compare interrupt flag \\ \hline
	\end{tabular}
	\caption{Registerbeschreibung – Capture-/Compare Register Timer B\\\Zitat[S. 375, Tab. 12-8]{ti:slau272d}}
	\label{tab:tb_ccc_register}
\end{table}



Wie bereits im Abschnitt \ref{Timer_CompareMode} zum Compare Mode beschrieben, legen die Output Mode Bits (OUTMOD) fest, wie der zugeh\"orige Ausgangspin bei einer \"ubereinstimmung des Z\"ahlerstandes mit dem Compare-Registerwert beeinflusst wird. Die Auswahl des passenden Output Mode ist entscheidend f\"ur die Erzeugung der gew\"unschten Ausgangssignale, wie beispielsweise bei der Pulsweitenmodulation.

Die Auswahl des ausl\"osenden Ereignisses f\"ur eine Capture- oder Compare-Operation wird ebenfalls \"uber Bits im TAxCCTLn- oder TBxCCTLn-Register gesteuert. F\"ur den Capture Mode wird hier beispielsweise mit dem CM-Bit festgelegt, ob die Erfassung bei einer steigenden, fallenden oder beiden Flanken des Eingangssignals erfolgen soll. Im Compare Mode definiert diese Einstellung, unter welchen Bedingungen die Vergleichsoperation als erfolgreich betrachtet wird und die entsprechende Aktion (Interrupt, Ausgangssignal\"anderung) ausgel\"ost wird. Dies kann beispielsweise ein reiner Vergleich oder auch ein Vergleich in Kombination mit dem \"uberlauf des Z\"ahlers im Up Mode sein. \Zitat[S. 351, Kap. 11.3.3 \& S. 375, Kap. 12.3.3, S. 292, Kap. 8.3.2]{ti:slau272d, davies:msp430}

Die sorgf\"altige Konfiguration dieser Einstellungen in den Capture/Compare-Registern ist unerl\"asslich, um den Timer pr\"azise an die Anforderungen der jeweiligen Applikation anzupassen.

Ein weiterer fundamentaler Aspekt der Timer-Konfiguration ist \ua die Wahl der Taktquelle, welche die Zeitbasis f\"ur den Z\"ahler und somit f\"ur alle zeitgesteuerten Operationen des Timers bestimmt. \AI

\newpage
\subsection{Timer Control-Register}
\label{TimerControlRegister}

Die Timer des MSP430FR5729 k\"onnen von verschiedenen internen Taktquellen getaktet werden, die jeweils unterschiedliche Eigenschaften und Anwendungsbereiche aufweisen. Die prim\"aren Taktquellen sind \Fachbegriff[Niederfrequente Taktquelle in Mikrocontroller-Systemen, die typischerweise von einem Quarzoszillator gespeist wird und f\"ur energiesparende Betriebsmodi verwendet wird.]{Auxiliary Clock} (\Abkuerzung{Auxiliary Clock}{ACLK}) und \Fachbegriff[Taktgesteuertes Signal, das typischerweise f\"ur Peripherieger\"ate verwendet wird und sich aus einer frei w\"ahlbaren Taktquelle ableiten l\"asst.]{Sub-Main Clock} (\Abkuerzung{Sub-Main Clock}{SMCLK}). Auch externe Taktquellen k\"onnen zur Taktung des Timers herangezogen werden wie \zB das TACLK/TBCLK-Register oder der INCLK-Pin. \Zitat[S. 71, Kap. 3.1, S. 163, Kap. 5.8 \& S. 289, Kap. 8.3.1]{ti:slau272d, davies:msp430}

Die Auswahl der Clock-Source f\"ur einen Timer erfolgt \"uber spezifische Bits im TAxCTL oder TBxCTL Timer Control Register. Das TASSEL-/TBSSEL-Bit legt fest, ob der Timer von TAxCLK/TBxCLK, ACLK, SMCLK oder INCLK getaktet wird. Die Wahl der Clock-Source hat einen direkten Einfluss auf die Timer-Frequenz, wobei die Timer-Frequenz nicht gleich der Frequenz der gew\"ahlten Clock-Source entsprechen muss. Durch optionale \Fachbegriff[Vorschaltglied in elektronischen Z\"ahlschaltungen oder Timern, welches die Frequenz eines Eingangssignals durch einen festen Faktor reduziert, um eine nachfolgende Verarbeitung mit geringerer Taktrate zu erm\"oglichen.]{Prescaler}-Werte wie dem ID-Bit und dem TAIDEX-/TBIDEX-Bit kann die Frequenz weiter individualisiert werden. \Zitat[S. 349, Kap. 11.3.1 \& S. 372, Kap. 12.3.1, S. 289, Kap. 8.3.1]{ti:slau272d, davies:msp430}

\begin{table}[h!]
	\small
	\centering
	\begin{tabular}{|c|l|c|c|p{8cm}|}
		\hline
		\textbf{Bit} & \textbf{Field} & \textbf{Type} & \textbf{Reset} & \textbf{Description} \\ \hline
		15 & Reserved & R & 0h & Reserved. Always reads as 0. \\ \hline
		14–13 & TBCLGRP & RW & 0h & \textbf{TBxCLn group:} Synchronously updates multiple Capture/Compare registers as needed. \\ \hline
		12–11 & CNTL & RW & 0h & Counter length \\ \hline
		10 & Reserved & R & 0h & Reserved. Always reads as 0. \\ \hline
		9–8 & TBSSEL & RW & 0h & clock source select \\ \hline
		7–6 & ID & RW & 0h & \textbf{Input divider:} together with TBIDEX divides the input clock \\ \hline
		5–4 & MC & RW & 0h & \textbf{Mode control:} \ref{Timer_CountMode} \\ \hline
		3 & Reserved & R & 0h & Reserved. Always reads as 0. \\ \hline
		2 & TBCLR & RW & 0h & Clears TBR and control logic. \\ \hline
		1 & TBIE & RW & 0h & Timer\_B interrupt enable \\ \hline
		0 & TBIFG & RW & 0h & Timer\_B interrupt flag \\ \hline
	\end{tabular}
	\caption{Registerbeschreibung – Control Register Timer B\\\Zitat[S. 372, Tab. 12-6]{ti:slau272d}}
	\label{tab:tb_c_register}
\end{table}

Die Timer-Frequenz bestimmt wiederum die Zeitbasis des Timers. Eine h\"ohere Timer-Frequenz f\"uhrt zu einer feineren Zeitaufl\"osung, da der Z\"ahler schneller inkrementiert wird. Dies erm\"oglicht pr\"azisere Zeitmessungen und die Erzeugung von Signalen mit h\"oherer Frequenz. Umgekehrt f\"uhrt eine niedrigere Frequenz zu einer gr\"oberen Zeitaufl\"osung, kann aber den Stromverbrauch reduzieren.

Ein weiteres Steuerbits wie das \NeuerBegriff{Mode Control-Bit (MC)} steuert die bereits in Kapitel \ref{Timer_CountMode} erl\"auterten Z\"ahl-Modi und das TAIE-/TBIE-Bit steuert, ob Interrupts Ein- oder Ausgeschaltet sind.

Die Auswahl der Clock-Source, des Prescalers und weiteren Steuerbits ist daher entscheidend, um die gew\"unschte Zeitbasis, Aufl\"osung und Verhalten f\"ur den zu konfigurierenden Timer zu erreichen um die Anforderungen der jeweiligen Anwendung optimal zu erf\"ullen.

\Tabelle{tb_c_register} fasst alle weiteren Register des Timers B mit ihren Beschreibungen zusammen. \AI

\subsection{Zusammenfassung und Einsatzm\"oglichkeiten}
\label{TimerEinsatzmoeglichkeiten}

Die detaillierte Auseinandersetzung mit der Timer-Architektur des MSP430FR5729 hat die Flexibilit\"at und Leistungsf\"ahigkeit dieser Peripheriekomponente verdeutlicht. Die Unterscheidung zwischen Timer des Typs A und B, die verschiedenen Betriebsarten (Count, Capture, Compare) sowie die vielf\"altigen Einstellm\"oglichkeiten der Capture/Compare-Register und die Auswahl der Taktquelle er\"offnen ein breites Spektrum an Anwendungsm\"oglichkeiten in eingebetteten Systemen.

Analog zur \"Ubersicht "What Timer Where?" von John H. Davies lassen sich die prim\"aren Einsatzgebiete der Timer des MSP430FR5729 wie folgt zusammenfassen: \Zitat[S. 356, Kap. 8.10]{davies:msp430}

\begin{itemize}
	\item \textbf{Zeitmessung und Zeitbasis:} Unabh\"angig vom Timer-Typ k\"onnen alle als eine pr\"azise Zeitbasis dienen. Durch die Wahl einer geeigneten Clock-Source und eines passenden Prescalers lassen sich genaue Zeitintervalle festlegen. Dies ist fundamental f\"ur das Zeitmanagement innerhalb des Mikrocontrollers und die Synchronisation mit externen als auch Internen Ereignissen. Timer A eignet sich hierbei oft f\"ur grundlegende Zeitsteuerungsaufgaben, w\"ahrend die flexiblere Konfigurierbarkeit des Timers vom Typ B wie \zB verschiedene Bit-L\"angen (\ref{TIMER&ISR}) eine feinere Anpassung an spezifische Zeitmessanforderungen erlaubt.

	\item \textbf{Ereigniserfassung (Capture):} Die Capture-Funktionalit\"at erm\"oglicht die pr\"azise Erfassung des Zeitpunkts externer Ereignisse. Dies ist unerl\"asslich f\"ur Anwendungen wie die Messung von Pulsweiten, die Frequenzmessung von Signalen oder die Erfassung der Ankunftszeit von Informationen in Kommunikationsprotokollen. Die M\"oglichkeit, sowohl steigende, fallende Flanken oder auch beide zu erfassen, erweitert den Anwendungsbereich in verschiedenen Szenarien deutlich.

	\item \textbf{Signalerzeugung (Compare/PWM):} Die Compare-Einheiten in Verbindung mit den verschiedenen Ausgangsmodi erlauben die Generierung pr\"aziser Ausgangssignale. Dies ist besonders relevant f\"ur die Pulsweitenmodulation, die zur Steuerung von Motoren, zur Dimmung von LEDs oder zur Erzeugung analog wirkender Signale eingesetzt wird. Der Up/Down Mode des Count-Modus in Kombination mit den Compare-Registern des Timer B bietet hierbei besonders flexible M\"oglichkeiten zur Erzeugung verschiedenster PWM-Signale.

	\item \textbf{Interrupt-Steuerung:} Sowohl Capture- als auch Compare-Ereignisse k\"onnen Interrupts ausl\"osen. Dies erm\"oglicht eine effiziente Reaktion des Mikrocontrollers auf zeitgesteuerte Ereignisse oder externe Signale, ohne die kontinuierliche abfrage des Timer-Status. Die pr\"azise Interrupt-Generierung tr\"agt maßgeblich zur Realisierung reaktiver und effizienter eingebetteter (Echtzeit-) Systeme bei.
\end{itemize}

Zusammenfassend l\"asst sich der grundlegende Aufbau eines Timers, vereinfacht nach dem Vorbild von Abbildung 8.5 und Abbildung 8.16 aus Davies' Buch, wie folgt darstellen:

Ein Timer besteht im Kern aus einem Z\"ahler (\ref{Timer_CountMode}), der durch eine ausgew\"ahlte Clock-Source (\ref{TimerControlRegister}) in definierten Schritten inkrementiert oder dekrementiert wird. Dieser Z\"ahler l\"auft gem\"a{\ss} der gew\"ahlten Betriebsart.

Zus\"atzlich verf\"ugt der Timer \"uber Sieben Capture/Compare-Kan\"ale. Jeder Kanal beinhaltet mindestens ein Capture/Compare-Register und eine zugeh\"orige Steuereinheit.

Im Capture Mode (\ref{Timer_CaptureMode}) wird der aktuelle Wert des Z\"ahlers in das CCRx-Register geschrieben, wenn ein durch die Steuereinheit ausgew\"ahltes Ereignis (\zB Flanke an einem Eingangspin) eintritt.

Im Compare Mode (\ref{Timer_CompareMode}) wird der aktuelle Wert des Z\"ahlers kontinuierlich mit dem Wert im CCRx-Register verglichen. Bei einer \"Ubereinstimmung l\"ost die Steuereinheit eine konfigurierte Aktion aus, wie beispielsweise das Setzen/R\"ucksetzen/Toggeln eines zugeh\"origen Ausgangspins oder die Generierung eines Interrupts, sofern dieser in der Steuereinheit aktiviert wurde.

Die Steuereinheit erm\"oglicht die Konfiguration des jeweiligen Kanals, einschließlich der Auswahl des Capture/Compare-Modus, des ausl\"osenden Ereignisses, des Ausgangsmodus und der Aktivierung/Deaktivierung des Interrupts.

\begin{figure}[h!]
	\centering
	\includegraphics[width=1.0\textwidth]{../Bilder/BlockDiagram_TimerB.png}
	\caption{Timer B Block \& Capture/Compare Channel 1\\\Zitat[S. 355, Kap. 8.16]{davies:msp430}}
	\label{fig:BlockDiagramm_Timer}
\end{figure}

Die Darstellung \Abbildung{BlockDiagramm_Timer} des Timer B als Block Diagram verbildlicht die grundlegenden Komponenten eines Timer-Kanals und deren Zusammenspiel. Die flexiblen Konfigurationsm\"oglichkeiten dieser einzelnen Bl\"ocke erm\"oglichen die Realisierung einer Vielzahl von Zeitsteuerungs- und Signalverarbeitungsaufgaben in eingebetteten Systemen mit dem MSP430FR5729.

\newpage
\section{Enhanced Universal Serial Communication Interface (eUSCI)}
\label{eUSCI}

Das eUSCI ist eine vielschichtige und flexible Serielle Peripheriekomponente des MSP430FR5729. Sie erm\"oglicht die Kommunikation mit externen Ger\"aten und Systemen \"uber eine Vielzahl an Schnittstellen und Protokollen. Auch essentielle Bausteine wie Sensoren, Aktoren und Speichermedien werden \"uber diese Art mit dem System verbunden. Dieses Kapitel beleuchtet die Architektur, verschiedene Betriebsmodi und Konfigurationsm\"oglichkeiten der Bereitgestellten Kommunikationstechnologien.

Der MSP430FR5729 verf\"ugt \"uber zwei Kommunikationskan\"ale welche in den folgenden Kapiteln n\"aher betrachtet werden. 

\subsection{\"Uberblick \"uber die eUSCI-Architektur}
\label{eUSCI_Architektur}

In der Theorie fu{\ss}t jede Form der Seriellen Kommunikation auf einem Taktgeber. Der zentrale Unterschied zwischen den jeweiligen Protokollen liegt darin, zu welchem Zeitpunkt der Taktgeber dem Sender gestattet, das n\"achste Bit auf einen Ausgangskanal zu schreiben, beziehungsweise dem Empf\"anger erm\"oglicht, das kommende Bit zu lesen. Dabei gibt es Synchrone und Asynchrone Ans\"atze, weshalb gleich zwei Kommunikationskan\"ale bereitgestellt werden. \Zitat[S. 494, Kap. 10]{davies:msp430}

Bei dem MSP430FR5729 ist der Kommunikationskanal vom Typ-B f\"ur Synchrone Daten\"ubertragung optimiert, w\"ahrend Kanal A vorrangig f\"ur asynchrone \"Ubertragungsverfahren vorgesehen ist. \Zitat[S. 496, Kap. 10.1.2]{davies:msp430}

Der wesentliche Unterschied zwischen diesen beiden \"Ubertragungsarten besteht darin, ob das Taktsignal ebenfalls mit \"ubertragen wird. Bei der synchronen Kommunikation, wie sie etwa mit den Protokollen SPI oder I$^{2}$C erfolgt, wird dieses Taktsignal explizit mitgef\"uhrt. Im Gegensatz dazu kommt das UART-Protokoll ohne ein separates Taktsignal aus da Sender und Empf\"anger \"uber eine gemeinsame Baudrate synchronisiert sind. \Zitat[S. 494, Kap. 10]{davies:msp430}

\newpage
Entsprechend ist der Univers\"alle Kommunikationskanal vom Typ B (\Abkuerzung{enhanced Universal Serial Communication Interface Type B}{eUSCI\_B}) speziell auf die Anforderungen synchroner Protokolle wie SPI und I$^{2}$C ausgelegt. Das \Abkuerzung{enhanced Universal Serial Communication Interface Type A}{eUSCI\_A}-Modul hingegen unterst\"utzt prim\"ar die UART-Kommunikation, kann jedoch dar\"uber hinaus auch eine asynchrone Variante des SPI-Protokolls abbilden. \Zitat[S. 496, Kap. 10.1.2]{davies:msp430}

\begin{table}[h!]
	\small
	\centering
	\begin{tabular}{|l|c|c|}
		\hline
		\textbf{Funktion} & \textbf{eUSCI\_A} & \textbf{eUSCI\_B} \\\hline
		UART (asynchron) & \checkmark & -- \\\hline
		SPI (synchron) & \checkmark (Master/Slave) & \checkmark (Master/Slave) \\\hline
		SPI (asynchron, nur TX) & \checkmark & -- \\\hline
		I\textsuperscript{2}C (synchron) & -- & \checkmark (Master/Slave) \\\hline
		LIN-kompatibel & \checkmark & -- \\\hline
		Automatische Baudratenerkennung (UART) & \checkmark & -- \\\hline
		Adress- und Broadcast-Modus (I\textsuperscript{2}C) & -- & \checkmark \\\hline
		Multimaster-Unterst\"utzung (I\textsuperscript{2}C) & -- & \checkmark \\\hline
	\end{tabular}
	\caption{Funktionsvergleich der eUSCI-Module des MSP430FR5729\\\Zitat[Kap. 18, 19, 20, S. 493, Kap. 10]{ti:slau272d, davies:msp430}}
	\label{tab:eusci-vergleich}
\end{table}

\Tabelle{eusci-vergleich} fasst nochmals alle Eigenschaften beider Kan\"ale zusammen und stellt sie gegen\"uber. Wobei Tiefgreifendere, Protokoll-Spezifische Funktionen in den entsprechenden Unterkapiteln n\"aher betrachtet werden.

\subsection{Einordnung vorhandener Kommunikationsschnittstellen}

Die vorliegende Arbeit befasst sich mit der Entwicklung einer interruptgesteuerten Benutzerschnittstelle. Die Evaluation einer daf\"ur geeigneten Schnittstelle zur Interaktion mit externen \Abkuerzung{Personal Computer}{PC} Systemen bestimmt ma{\ss}geblich den weiteren verlauf der Evaluation und des Projekts. Daher ist eine einf\"uhrende Einordnung der zur Verf\"ugung stehenden seriellen Protokollen erforderlich, um im weiteren Verlauf gezielt auf jene Technologie eingehen zu k\"onnen, die im Kontext der Arbeit von praktischer Relevanz ist.

Die Kommunikation \"uber synchrone Protokolle wie I$^{2}$C und SPI eignen sich besonders gut f\"ur den Datenaustausch zwischen einem Microcontroller und seinen Peripherieger\"aten oder weiteren Microcontrollern im Master-Slave-Verh\"altnis. Welche der beiden Technologien im jeweiligen Anwendungsfall zum Einsatz kommt, h\"angt unter anderem von der Anzahl der beteiligten Ger\"ate sowie der Distanz zu den Kommunikationspartnern ab. Weitere technische Unterschiede dieser Protokolle sind in \Tabelle{synchrone_protokolle} aufgelistet.

\begin{table}[h!]
	\small
	\centering
	\begin{tabular}{|p{4.5cm}|p{4.5cm}|p{4.5cm}|}
		\hline
		\textbf{Kriterium} & \textbf{SPI} & \textbf{I$^{2}$C} \\\hline
		\textbf{Signalleitungen} & 4 Leitungen: SCLK, MOSI, MISO, CS (pro Slave) & 2 Leitungen: SCL (Takt), SDA (Daten) \\\hline
		\textbf{Adressierung} & Keine; Slaves \"uber eigene Chip Selects (CS) & Ja; \"uber 7- oder 10-Bit-Adresse auf dem Bus \\\hline
		\textbf{Daten\"ubertragung} & \Fachbegriff[Gleichzeitige Daten\"ubertragung in beide Richtungen]{Vollduplex} m\"oglich & \Fachbegriff[Daten\"ubertragung zu einem Zeitpunkt nur in eine Richtung m\"oglich]{Halbduplex} \\\hline
		\textbf{Taktfrequenz} & Bis > 10,MHz (ger\"ateabh\"angig) & Typisch 100,kHz, 400,kHz, bis 3.4,MHz (High-Speed) \\\hline
		\textbf{Komplexit\"at des Protokolls} & Einfach, ohne Start-/Stopp- oder ACK-Signale & H\"oher, mit Start-/Stoppbedingungen und Acknowledgements \\\hline
		\textbf{Multimaster-Unterst\"utzung} & Nein (standardm\"aßig) & Ja \\\hline
		\textbf{Skalierbarkeit (Anzahl Ger\"ate)} & Eingeschr\"ankt, abh\"angig von verf\"ugbaren CS-Leitungen & Hoch, bis zu 128 Ger\"ate durch Adressierung \\\hline
		\textbf{Typische Einsatzgebiete} & Hochgeschwindigkeits-kommunikation (\zB SD-Karten, Displays) & Niedriggeschwindigkeits-Komponenten (\zB Sensoren, EEPROMs) \\\hline
		\textbf{Leitungsl\"ange / St\"oranf\"alligkeit} & Gut f\"ur kurze, direkte Verbindungen & H\"ohere Anf\"alligkeit f\"ur St\"orungen und Begrenzung durch Leitungskapazit\"at \\\hline
	\end{tabular}
	\caption{Vergleich der synchronen seriellen Protokolle SPI und I$^{2}$C\\\Zitat[S. 497, Kap. 10.2, S. 534, Kap. 10.7]{davies:msp430}}
	\label{tab:synchrone_protokolle}
\end{table}

Zusammenfassend l\"asst sich feststellen, dass sich die synchrone Daten\"ubertragung nur bedingt f\"ur die Kommunikation mit einem auf Windows oder Linux basierenden System eignet. Im Gegensatz dazu sind asynchrone Protokolle wie UART f\"ur diese Art der Anwendung deutlich besser geeignet. 

Das UART-Protokoll zeichnet sich nicht nur durch eine einfache Implementierung aus, sondern ist auch \"au{\ss}erst robust und ressourcenschonend -Eigenschaften, die insbesondere bei modularen, \Fachbegriff[Automatische Erkennung und Integration von Komponenten in ein System ohne manuelle Konfiguration]{Plug-and-Play}-f\"ahigen Systemkomponenten entscheidend sind. Der Verzicht auf eine gemeinsame Taktleitung erlaubt eine Punkt-zu-Punkt-Verbindung mit vergleichsweise geringen Hardwareanforderungen. 

Diese Vorteile gelten gleicherma{\ss}en f\"ur die gegen\"uberliegende Seite der Schnittstelle: Alle g\"angigen Betriebssysteme wie Windows, Lunix oder macOS stellen standardm\"aßig Treiber f\"u die asynchrone serielle Daten\"ubertragung bereit. Unter Windows erfolgt dies beispielsweise \"uber sogenannte \NeuerBegriff{COM-Ports}, w\"ahrend unter Linux Schnittstellen wie \NeuerBegriff{/dev/ttySx} oder \NeuerBegriff{/dev/ttyUSBx} verwendet werden.

Im falle des MSP430FR5729 erfolgt die UART-Kommunikation zus\"atzlich interruptgesteuert. Dies erlaubt es, eingehende Daten \"uber eine eigens definierte ISR zu verarbeiten, was eine latenzarme, gleichzeitig jedoch energieeffiziente Verarbeitung erm\"oglicht.\Zitat[S. 574, Kap. 10.12]{davies:msp430}

Aus diesen Gr\"unden stellt UART die technisch sinnvollste Wahl f\"ur die Kommunikation zwischen dem MSP430FR5729 und einem PC mit Windows, Linux oder macOS dar. Das Protokoll erlaubt eine minimalinvasive, betriebssystemkompatible und energieeffiziente Verbindung. Im weiteren Verlauf wird die asynchrone universelle serielle Schnittstelle mit dem UART Protokoll n\"aher elaboriert.\AI

\subsection{eUSCI\_A Modul: UART-Modus (Asynchrone Kommunikation)}
\label{eUSCI_UART}

F\"ur ein tiefgreifendes Verst\"andnis des Zusammenspiels zwischen dem Interface und dem UART-Protokoll ist es unerl\"asslich, zun\"achst die fundamentalen technischen Aspekte, notwendige Register und charakteristische Merkmale zu erl\"autern.

\subsubsection{Informations\"ubertragung}
\label{UART_uebertragung}

Die Baudrate, wie in Kapitel \ref{eUSCI_Architektur} bereits er\"ortert, fungiert als entscheidender Synchronisationsmechanismus f\"ur asynchrone Daten\"ubertragungen. Dies impliziert, dass Sender und Empf\"anger sich zwar nicht an ein pr\"azises Timing f\"ur die \"ubertragung einzelner Bits halten m\"ussen, jedoch eine \"Ubereinstimmung hinsichtlich der Frequenz f\"ur die \"Ubertragung ganzer Bl\"ocke (typischerweise Bytes oder Zeichen) erforderlich ist. \Abbildung{uart_send} visualisiert beispielhaft die \"ubertragung zweier Bl\"ocke, die jeweils durch ein Start-Bit (\Abkuerzung{Start-Bit}{ST}) eingeleitet und durch ein Stop-Bit (\Abkuerzung{Stop-Bit}{SP}) abgeschlossen werden. Durch die Verwendung dieser Rahmenbits ergibt sich bei einer Konfiguration von acht Datenbits eine Netto-Datenrate von 8/10 der Brutto-\"Ubertragungsrate. Das bedeutet dass von zehn \"ubertragenen Bits acht Bits die eigentliche Nutzinformation darstellt. Es gibt auch die M\"oglichkeit, flexibel auf unterschiedliche Baudraten zu reagieren. \"Uber die Automatische Baudraten-Erkennung, erl\"autert in \ref{auto_baud}, kann mit mehreren Kommunikationspartnern unterschiedlicher Baudraten Kommuniziert werden.

\begin{figure}[h!]
	\centering
	\includegraphics[width=1.0\textwidth]{../Bilder/Baudrate.png}
	\caption{UART \"ubertragung der Werte 0x55 und 0xFF\\\Zitat[S. 576, Abb. 10.18]{davies:msp430}}
	\label{fig:uart_send}
\end{figure}

Die einzelnen Bits innerhalb eines Datenblocks werden mittels des \Fachbegriff[Bin\"ares Leitungscodierungsverfahren, bei dem der Signalpegel w\"ahrend eines Bitintervalls konstant bleibt und nicht zwischen den Bits auf einen Nullpegel zur\"uckkehrt]{non-return to zero} (\Abkuerzung{non-return to zero}{NRZ}) verfahren kodiert und \"ubertragen. Eine Typische Baudrate f\"ur eingebettete Systeme betr\"agt 9600 Baud, obgleich auch h\"ohere Frequenzen zur beschleunigten Daten\"ubertragung Anwendung finden k\"onnen.

Die physikalische Verbindung zweier Parteien wird \"ublicherweise \"uber drei Leitungen realisiert. Eine Leitung f\"ur jede Kommunikationsrichtung (\Abkuerzung{Transmit Data}{TxD} zu \Abkuerzung{Receive Data}{RxD}) und eine f\"ur die gemeinsame Masse. Dies erm\"oglicht eine \NeuerBegriff{Vollduplex-Kommunikation}, bei der beide Seiten gleichzeitig und unabh\"angig voneinander Daten Senden und Empfangen k\"onnen. Voraussetzungen hierf\"ur sind separate Sende- und Empfang-Schieberegister sowie dedizierte Puffer (\Fachbegriff[Empfangspuffer f\"ur UART Kommunikation]{UCAxRXBUF} und \Fachbegriff[Sendepuffer f\"ur UART Kommunikation]{UCAxTXBUF}) f\"ur beide Kommunikationsrichtungen in der Hardware des Interfaces. \Zitat[S. 499, Kap. 18.4.6 \& 18.4.7]{ti:slau272d}

\newpage
Der Typische Ablauf beim Empfang eines Blocks \"uber UART gestaltet sich wie folgt:

\begin{enumerate}
	\item Beginn der Zeitmessung mit der fallender Flanke, die das Startbit einleitet.
	\item Abtastung des Eingangs nach einer halben Bitperiode zur Best\"atigung eines g\"ultigen Startbits.
	\item Weitere Abtastung nach einer vollst\"andigen Bitperiode zur Erfassung des ersten Datenbits (LSB).
	\item Wiederholung dieses Vorgangs f\"ur alle 8 Datenbits bis zum h\"ochstwertigen Bit (MSB).
	\item Abschlie{\ss}ende Abtastung nach einer weiteren Bitperiode zur \"Uberpr\"ufung des Stopbits (High-Pegel erwartet). Liegt stattdessen ein Low-Pegel vor, wird ein Framing-Fehler erkannt.
\end{enumerate}

\Abbildung{uart_uebertragung} visualisiert diesen Empfangsprozess unter Verwendung einer sogenannten \Fachbegriff[Bestimmt den Zeitpunkt der Abtastung eingehender Bits; sie wird \"ublicherweise aus einer \"ubergeordneten Taktquelle (\zB SMCLK) abgeleitet und beeinflusst ma{\ss}geblich die Genauigkeit der Daten\"ubertragung.]{sampling clock}. Diese Abtastfrequenz ist \"ublicherweise um den Faktor 16 h\"oher als die konfigurierte Baudrate. Das Oversampling (\"Ubertastung) ist notwendig um das eintreffende Start-Bit zuverl\"assig und zeitnah auch zwischen den regul\"aren Bit-Takten detetktieren zu k\"onnen.

\begin{figure}[h!]
	\centering
	\includegraphics[width=1.0\textwidth]{../Bilder/uart_protocoll.png}
	\caption{UART \"ubertragung der Werte 0x55 und 0xFF\\\Zitat[S. 577, Abb. 10.19]{davies:msp430}}
	\label{fig:uart_uebertragung}
\end{figure}

Die interne \NeuerBegriff{Bit-Clock} (\Abkuerzung{Bit-Clock}{BITCLK}) des Empf\"angers wird mit der fallenden Flanke des eingegangenen Start-Bits synchronisiert und operiert mit der Frequenz der eingestellten Baudrate. Da die fallende Flanke des Start-Bits zu einem beliebigen Zeitpunkt relativ zur Sampling Clock auftreten kann, entsteht ein initialer Synchronisationsfehler von bis zu einer halben Periode der Sampling Clock. Die in \Abbildung{uart_uebertragung} dargestellten Szenarien, bezeichnet als \"Input 1\" und \"Input 2\", illustrieren die hieraus resultierende minimale und maximale zeitliche Verschiebung bei der Detektion der Startbit-Flanke, abh\"angig vom Phasenverh\"altnis zwischen dem Datensignal und der Sampling Clock.\Zitat[S. 476, kap. 18.2, S. 574, Kap. 10.12 \& S. 575, Kap. 10.12.1]{ti:slau272d, davies:msp430}

\subsubsection{Datenintegrit\"at, Fehlererkennung und weitere technischen Details}
\label{datenintegritaet}

Zur Sicherstellung der Datenintegrit\"at kann eine Fehlererkennung, beispielsweise \"uber ein Parit\"atsbit, eingesetzt werden. Ein UART-Datenpaket (Frame) besteht somit typischerweise aus einem Start-Bit, sieben oder acht Datenbits, optional einem Parit\"atsbit (konfigurierbar f\"ur gerade oder ungerate Parit\"at) und einem oder (seltener) mehreren Stop-Bits. 

Die Implementierung komplexerer Fehlererkennung oder gar Fehlerkorrekturmechanismen, wie \zB Pr\"ufsummen (Checksum), obliegt \"ublicherweise der \"ubergeordneten Protokollebene, die auf der UART-Kommunikation aufsetzt. Dar\"uber hinaus ist f\"ur eine erfolgreiche Kommunikation die eindeutige Festlegung der Bitreihenfolge essentiell. Die \Abkuerzung{Least Significant Bit}{LSB}-first-Konvention ist De-facto-Standard.\Zitat[S. 574, Kap. 10.12 \& S. 575, Kap. 10.12.1]{davies:msp430}

Die Automatische Fehlererkennung des Interfaces erlaubt es dem Benutzer, schnell und ohne gron{\ss}en Implementierungsaufwand auf Grenzf\"alle und \"Ubertragungsfehler zu reagieren. \Tabelle{uart_error_flags} schl\"usselt alle wichtigen Fehler-Flags, mit ihren zugeh\"origen Beschreibungen auf.

\begin{table}[h!]
	\small
	\centering
	\begin{tabular}{|l|c|p{8.5cm}|}
		\hline
		\textbf{Fehlerbedingung} & \textbf{Fehler-Flag} & \textbf{Beschreibung} \\
		\hline
		Framing-Fehler & UCFE & Tritt auf, wenn das Stoppbit nicht den erwarteten High-Pegel hat. Bei zwei Stoppbits werden beide gepr\"uft. Bei Fehler wird das UCFE-Bit gesetzt. \\\hline
		Parit\"atsfehler & UCPE & Entsteht durch eine Abweichung zwischen berechneter und tats\"achlicher Parit\"at. Adressbits werden in die Berechnung einbezogen. Bei Fehler wird UCPE gesetzt. \\\hline
		Empfangs\"uberlauf & UCOE & Wenn ein neues Zeichen empfangen wird, bevor das vorherige gelesen wurde, wird ein \"Uberlauf erkannt und das UCOE-Bit gesetzt. \\\hline
		Break-Bedingung & UCBRK & Wird erkannt, wenn alle Bits (Daten-, Parit\"ats- und Stoppbits) auf Low liegen (bei deaktivierter Baudratenerkennung). UCBRK wird gesetzt und \ggf auch UCRXIFG, wenn UCBRKIE aktiv ist. \\\hline
	\end{tabular}
	\caption{UART-Fehlerbedingungen und zugeh\"orige Status-Flags des MSP430FR5729\\\Zitat[S. 483, Tab. 18-1]{ti:slau272d}}
	\label{tab:uart_error_flags}
\end{table}

\newpage
Weitere technische Spezifikationen sind in \Tabelle{uart_features} zusammengefasst. Eine minnimale Konfiguration der Schnittstelle auf den UART-Betrieb wird in \ref{eUSCI_Konfiguration} detailliert beschrieben.

\begin{table}[h!]
	\small
	\centering
	\begin{tabular}{|p{6.5cm}|p{7cm}|}
		\hline
		\textbf{Funktion} & \textbf{Beschreibung} \\
		\hline
		Multiprozessor-Kommunikationsprotokolle & Unterst\"utzt integrierte Idle-Line- und Address-Bit-Protokolle f\"ur Kommunikation in Multiprozessorsystemen \\
		\hline
		Energiesparmodus-Unterst\"utzung & Startflankenerkennung (Start Edge Detection) im Empf\"anger erm\"oglicht automatisches Aufwachen aus LPMx-Modi (ausgenommen LPMx.5) \\
		\hline
		Fehlererkennung & Statusflags zur Detektion und Unterdr\"uckung von Kommunikationsfehlern (z.\,B. Framing-, Parit\"ats- oder \"uberlauffehler) \\
		\hline
		Adresserkennung & Statusflags zur Erkennung von adressierten Datenpaketen in multiprozessorf\"ahigen Systemen \\
		\hline
		Interrupt-Unterst\"utzung & Unabh\"angige Interruptquellen f\"ur Empfang, \"ubertragung, Startbit-Empfang sowie Abschluss der \"ubertragung \\
		\hline
	\end{tabular}
	\caption{Technische Merkmale der UART-Schnittstelle des MSP430FR5729\\\Zitat[S. 476, kap. 18.2]{ti:slau272d}}
	\label{tab:uart_features}
\end{table}

\newpage
\subsubsection{Automatische Baudraten-Erkennung}
\label{auto_baud}

Neben der Einrichtung einer statischen Baudrate kann die automatische Baudraten-Erkennung selbstst\"andig, \"uber eine \NeuerBegriff{Break/Sync Sequenz}, die vom Sender verwendete Baudrate ermitteln. Diese Synchronisations-Sequenz besteht aus einem \NeuerBegriff{Break} und einem \NeuerBegriff{Sync} Feld. Der Bereich der erkennbaren Baudraten liegt, im Oversampling-Modus zwischen 244 Baud (im niedrigfrequenz-Modus beginnend ab 15 Baud) und einem Megabaud. Ein Break beinhaltet zwischen 11 und 21 \"ubertragenen 0en, w\"ahrenddessen alle weiteren empfangenen 0en einen \NeuerBegriff{Break Timeout}-Fehler ausl\"osen. Aus Konformit\"atsgr\"unden sollte das UART Protokoll auf acht Datenbits, mit LSB first, keiner Parit\"at und einem Stop-Bit konfiguriert werden. In \Abbildung{auto_baud} ist die beschriebene Break/Sync-Sequenz dargestellt.

\begin{figure}[h!]
	\centering
	\includegraphics[width=1.0\textwidth]{../Bilder/auto_baud.png}
	\caption{Automatische Baudraten-Erkennung - Break/Sync Sequenz\\\Zitat[S. 481, Abb. 18-5]{ti:slau272d}}
	\label{fig:auto_baud}
\end{figure}

Der Synchronisations-Prozess beginnt mit der \"Ubertragung des Hexadezimalen Werts 55. Die Zeit zwischen der ersten und letzten fallenden Flanke wird gemessen, um die vom Sender verwendete Baudrate zu ermitteln. Grafisch dargestellt in \Abbildung{sync_field}. Falls die m\"oglich messbare Zeit \"uberschritten wird erscheint ein \NeuerBegriff{Sync Timeout}-Fehler. Falls die Messung erfolgreich war, kann nach dem setzen des \NeuerBegriff{Receive Interrupt Flags} die Information ausgelesen werden. 

Nach jedem empfangenen Zeichen ist zu beachten, das \NeuerBegriff{UCDORM}-Bit zur\"uckzusetzen. Ist dieses Bit gesetzt, werden alle Zeichen empfangen, nicht aber in das Puffer-Register der Schnittstelle geschrieben.\Zitat[S. 481, Kap. 18.3.4]{ti:slau272d}

\begin{figure}[h!]
	\centering
	\includegraphics[width=1.0\textwidth]{../Bilder/sync_field.png}
	\caption{Automatische Baudraten-Erkennung - Sync Feld\\\Zitat[S. 481, Abb. 18-6]{ti:slau272d}}
	\label{fig:sync_field}
\end{figure}

\newpage
\subsection{eUSCI-Konfiguration}
\label{eUSCI_Konfiguration}

Die Initialisierung und Konfiguration des eUSCI\_A-Moduls f\"ur den UART-Betrieb erfordert eine genaue Abfolge von Schritten zum setzen von verschiedenen Bits und Werten in die daf\"ur vorgesehene Register. Der Prozess beginnt mit dem setzen des \Code{UCSWRST}-Bits. Dieses Bit erm\"oglicht die Konfiguration der Schnittstelle und schließt unerw\"unschtes verhalten aus. Dabei wird das \Code{UCTXIFG}-Bit, zum freigeben der Konfiguration, gesetzt. Zudem werden diverse Interrupt-Enable-Bits wie \Code{UCRXIE} und \Code{UCTXIE} sowie Status- und Fehlerflags (\Code{UCRXIFG, UCRXERR, UCBRK, UCPE, UCOE, UCFE, UCSTOE, UCBTOE}) im \Code{UCAxSTATW}-Register und \Code{UCAxIFG}-Register gel\"oscht oder in einen definierten Anfangszustand gebracht. Dies versetzt das eUSCI\_A-Modul in einen sicheren Reset-Zustand.\Zitat[S. 478, Kap. 18.3.1]{ti:slau272d}

Nachdem das Modul sicher im Reset-Zustand initialisiert wurde, k\"onnen weitere spezifische Konfigurationsparameter f\"ur den UART-Betrieb gesetzt werden. Hierzu z\"ahlen insbesondere die folgenden Kontrollbits im \Code{UCAxCTLWn}-Register, welche die grundlegenden Betriebscharakteristika des UART-Modus definieren:

\begin{itemize}
	\item \textbf{UCPEN:} Aktiviert die Parit\"atspr\"ufung.
	\item \textbf{UCPAR:} Festlegen einer geraden oder ungeraden Parit\"at.
	\item \textbf{UCMSB:} LSB oder MSB-first.
	\item \textbf{UC7BIT:} Konfiguriert die Datenl\"ange auf 7 oder 8-Bit.
	\item \textbf{UCSPB:} Anzahl der Stop-Bits.
	\item \textbf{UCMODEx:} W\"ahlt den UART-Modus. (\zB 0 f\"ur Normal-Betrieb, 3 f\"ur  automatische Baudratenerkennung)
	\item \textbf{UCSYNC:} F\"ur den asynchronen UART-Betrieb muss dieses Bit auf 0 gesetzt werden.
	\item \textbf{UCSSELx:} Taktquelle f\"ur Baudratengenerator.
\end{itemize}

\"Uber die genannten grundlegenden Einstellungen hinaus existieren weitere spezifischere Kontrollbits. Beispielsweise ist das \Code{UCRXEIE}-Bit zum freigeben des Fehler-Interrupts und das \Code{UCBRKIE} zum aktivieren der Break-Interrupts zust\"andig. Spezialfunktionen wie der Multiprozessor-Modus, welcher \"uber das \Code{UCTXADDR}-Bit gesteuert wird, oder das Senden eines Break-Zeichens mittels \Code{UCTXBRK} sind f\"ur eine Standard-UART-Konfiguration oft zu vernachl\"assigen, sofern diese Funktionalit\"aten nicht explizit gefordert sind.\Zitat[S. 495, Kap. 18.4.1 \& S. 496, Kap. 18.4.2]{ti:slau272d}

Eine weitere essenzielle Konfiguration f\"ur den UART-Betrieb ist die der Baudrate. Diese erfolgt \"uber das \Code{UCAxBRW}-Register und das \Code{UCAxMCTLW}-Register. Die korrekte Wertermittlung f\"ur diese Register ist direkt von der Frequenz der zuvor mittels \Code{UCSSELx} gew\"ahlten Taktquelle sowie der angestrebten Baudrate abh\"angig. Der Family User's Guide des MSP430FR5729 von Texas Intruments liefert, f\"ur die Berechnung dieser Werte, detaillierte Formeln und Beispieltabellen. Das \Code{UCAxBRW}-Register nimmt den ganzzahligen Anteil des Baudratenteilers (Prescaler) auf. Das \Code{UCAxMCTLW}-Register beinhaltet die Konfiguration f\"ur die Modulation der Frequenz, sowie das Bit zum aktivieren des Oversampling-Modus. Durch das \Code{UCBRFx}-Bit wird, in der ersten Modulationsstufe, die Feineinstellung des Prescalers vorgenommen. In der zweiten Modulationsstufe wird durch \Code{UCBRSx} ein Modulationsmuster f\"ur die BITCLK festgelegt. Im letzten Bit-Feld kann nun der Oversampling-Modus aktiviert oder deaktiviert werden.\Zitat[S. 487, Kap. 18.3.10 \& S. 497, Kap. 18.4.3, 18.4.4]{ti:slau272d}

Obwohl weitere spezifische Einstellungen m\"oglich sind, w\"urde deren detaillierte Er\"orterung den Rahmen dieser \"ubersicht \"uberschreiten. Eine unerl\"assliche, abschließende Konfigurationsmaßnahme vor der Inbetriebnahme betrifft jedoch die Port-Pins: Die f\"ur den UART Betrieb verwendeten Pins m\"ussen, \"uber die Function-Select-Register, auf die asynchrone UART-Kommunikation konfiguriert werden. \Zitat[S. 294, Kap. 8.2.5]{ti:slau272d}

Nach Abschluss aller Konfigurationseinstellungen wird das \Code{UCSWRST}-Bit im \Code{UCAxCTLW0}-Register gel\"oscht (auf 0 zur\"uckgesetzt). Dieser Schritt hebt den Reset-Zustand auf und aktiviert das eUSCI-Modul mit der zuvor definierten Konfiguration. Optional k\"onnen nun die gew\"unschten Interrupts, wie \zB der Sende- (\Code{UCTXIE}), Empfangs- (\Code{UCRXIE}), Transmit-Complete- (\Code{UCTXCPTIE}) oder Start-Bit-Interrupt (\Code{UCSTTIE}), im \Code{UCAxIE}-Register aktiviert werden, um eine ereignisgesteuerte Datenverarbeitung zu erm\"oglichen. \Zitat[S. 502, Kap. 18.4.10]{ti:slau272d}
	
Diese sorgf\"altige Konfigurationssequenz ist entscheidend f\"ur die zuverl\"assige Funktion der UART-Schnittstelle des MSP430-Mikrocontrollers.

\subsection{Zusammenfassung}
\label{eUSCI_Zusammenfassung}

Ein detailliertes Blockdiagram des eUSCI\_A-Moduls, konfiguriert f\"ur den UART-Betrieb ist in \Abbildung{BlockDiagramm_eUSCI_A_UART} dargestellt.

\begin{figure}[h!]
	\centering
	\includegraphics[width=1.0\textwidth]{../Bilder/eUSCI_UART.png}
	\caption{eUSCI Typ A - UART-Modus\\\Zitat[S. 477, Kap. 18.2]{ti:slau272d}}
	\label{fig:BlockDiagramm_eUSCI_A_UART}
\end{figure}