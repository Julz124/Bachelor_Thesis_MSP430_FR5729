% !TEX encoding = UTF-8 Unicode
% !TEX root =  Bachelorarbeit.tex

\section{Verwendete Hilfsmittel}

\subsection{Erkl\"arung zur Nutzung von KI-Sprachmodellen zur stilistischen \"Uberarbeitung}
Im Rahmen dieser Bachelorarbeit wurden \NeuerBegriff{Large Language Models}, namentlich ChatGPT-4o und GeminiAI 2.5 Flash, zur kritischen Bewertung und konstruktiven \"Uberarbeitung der sprachlichen Ausdrucksweise eingesetzt. Dies umfasst stilistische und Grammatikalische Optimierung in folgenden Bereichen:

\begin{itemize}
	\item Verbesserung der Lesbarkeit und sprachlichen Klarheit,
	\item Vereinheitlichung des wissenschaftlichen Ausdrucks,
	\item Korrekturvorschl\"age von Grammatik-, Rechtschreibung- und Zeichensetzung.
\end{itemize}

Alle Kapitel, die \"uberarbeitet werden, sind am Ende mit einem Hinweis in einer Fu{\ss}note gekennzeichnet.

Die inhaltliche Verantwortung f\"ur alle Aussagen, Argumentationen und wissenschaftlichen Schlussfolgerungen liegt vollst\"andig bei dem Autor. Die genannten KI-Modelle werden \textbf{nicht} zur Generierung fachlicher Inhalte, zur Datenanalyse oder zur Strukturierung argumentativer Abschnitte verwendet.

Die Verwendung dieser Werkzeuge erfolgt unter Beachtung geltender ethischer Richtlinien f\"ur wissenschaftliche Arbeiten sowie der Anforderung an Eigenst\"andigkeit und Transparenz.

\newpage
\subsection{Erkl\"arung zur Erstellung von Diagrammen}

Zur Erstellung von Diagrammen wird das Tool \textit{PlantUML} (verf\"ugbar unter \url{https://www.plantuml.com}) verwendet. 

Mit Hilfe des Tools werden verschiedene UML-Diagramme wie Aktivit\"atsdiagramme, Klassendiagramme und Zustandsdiagramme (State-Machine-Diagramme) modelliert. PlantUML erm\"oglicht die textuelle Beschreibung von Diagrammen, welche anschlie{\ss}end automatisch als grafische Darstellung generiert werden. Dies erleichtert sowohl die Versionierung als auch die Nachvollziehbarkeit der Diagrammerstellung im Entwicklungsprozess.