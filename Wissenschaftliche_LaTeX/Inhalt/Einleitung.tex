% !TEX encoding = UTF-8 Unicode
% !TEX root =  ../Bachelorarbeit.tex

\chapter{Einleitung}
\label{cha:Einleitung}

\section{Das Ziel dieser Arbeit}
\label{sec:ZielDerArbeit}

Diese Bachelor-Thesis befasst sich mit der Entwicklung einer \Fachbegriff{Ein Mechanismus zur ereignisorientierten Unterbrechung des normalen Programmablaufs.}{interruptgesteuerten}[interruptgesteuert] Benutzerschnittstelle auf einem \Fachbegriff{Ein Mikrocontroller, der f\"ur energieeffiziente Anwendungen optimiert ist. Typischerweise eingesetzt in batteriebetriebenen Embedded-Systemen.}{Low-Power-Mikrocontroller}, zur \"Uberwachung und Manipulation von \Fachbegriff{Speicherzellen, die fl\"uchtig sind und ihre Inhalte beim Ausschalten verlieren. \Zitat{muller2012rechnerarchitektur}}{Registern}[Register] und Speicherzellen in \Abkuerzung{Random Access Memory}{RAM} und \Abkuerzung{Ferroelectric Random Access Memory}{FRAM}.

Im Zuge des Arbeitsauftrags entwickelt diese Arbeit ein unabh\"angiges \Fachbegriff{Eine funktionale Einheit innerhalb eines gr\"o{\ss}eren Systems, die separat entwickelt und gewartet werden kann.}{Modul}, das sich bei Bedarf aktivieren oder deaktivieren l\"asst.

\section{Die Umgebung, in der die Arbeit entstand}
\label{sec:EntstehungsUmgebungArbeit}

Die Entwicklung der Software geschieht in Absprache mit Herrn Prof. Dr. Iren\"aus Schoppa, der ein zus\"atzliches \Fachbegriff{Ein Softwarewerkzeug, das spezifische Aufgaben oder Funktionen innerhalb eines gr\"o{\ss}eren Systems oder Workflows erf\"ullt.}{Tool} zur tiefgreifenden Analyse unter Realbedingungen von Register- und Speicherinhalten f\"ur Studenten im Praktikum von Mikroprozessorsysteme ben\"otigt.

Als Entwicklungsbasis kommt der in \Abkuerzung{Mikroprozessorsysteme}{MIPS} herangezogene MSP430FR5729 von Texas Instruments zum Einsatz, welcher bereits ein ausgereifter und etablierter \Abkuerzung{Low-Power-Mikrocontroller}{LP-MCU} im industriellen Umfeld ist. Der Fokus dieser Arbeit liegt auf den f\"ur die Implementierung wesentlichen Aspekten. Zahlreiche dem Prozessor zugrundeliegende Technologien werden aus diesem Grund bewusst ausgeklammert. \Zitat{davies:msp430}

\newpage
\section{Der Inhalt dieser Arbeit}
\label{sec:AufbauDieserArbeit}

\begin{description}

	\item[Grundlagen, Technologien und Evaluation:] \Kapitel{UeberblickEntwicklungsplattform} erarbeitet die f\"ur die Planung und Umsetzung verwendeten Grundlagen und Technologien. Das Kapitel erkl\"art die grundlegenden Eigenschaften und den Aufbau des Softwareentwicklungsprozesses auf einem LP-MCU und referenziert die ben\"otigten Dokumentationen.
	
	\item[Konzeptionierung der Benutzerschnittstelle:] Dieses Kapitel umfasst die Dokumentation der gesamten Planungsphase des Observer-Moduls. Hier entsteht eine \"Ubersicht \"uber die L\"osungsfindung, und die zur Planung erforderlichen Dokumente und Diagramme.
	
	\item[Die Entwicklung des Observers:] Im folgenden steht die Dokumentation der tats\"achlichen Programmierung des Software-Moduls im Vordergrund. Dieser Teil kl\"art die Implementierung und arbeitet den Entwicklungsverlauf anhand von Beispielen schrittweise ab.
	
	\item[Fazit und kritische Bewertung:] Das Fazit fasst die gemachten Erfahrungen sowie die Ergebnisse der Planung und Entwicklung abschlie{\ss}end zusammen und bewertet sie kritisch. Zus\"atzlich bietet das Kapitel einen kurzen Ausblick auf Erweiterungsm\"oglichkeiten und potenzielle Optimierungsschritte.

\end{description}

\section{Viele Informationen, wenige Quellen \dots}
\label{sec:Quellenlage}

Grunds\"atzlich ist es relativ einfach professionelle Quellen zu den Themen rund um den MSP430 und dessen integrierten Technologien zu finden, da -- wie bereits erw\"ahnt -- der MSP430 weitestgehend im industriellen Umfeld etabliert ist. Der Hersteller -- Texas Instruments -- stellt dazu ausreichend viele Informationen zur Softwareentwicklung zur verf\"ugung, was die Notwendigkeit vieler weiterer Quellen hinf\"allig macht.