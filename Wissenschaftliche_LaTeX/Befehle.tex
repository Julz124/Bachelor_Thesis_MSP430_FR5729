% !TEX encoding = UTF-8 Unicode
% !TEX root =  Bachelorarbeit.tex
% Eigene Befehle und typographische Auszeichnungen für diese


% einfaches Wechseln der Schrift, z.B.: \changefont{cmss}{sbc}{n} ---------------------------------------
\newcommand{\changefont}[3]{\fontfamily{#1} \fontseries{#2} \fontshape{#3} \selectfont}


% Abkürzungen mit korrektem Leerraum --------------------------------------------------------------------
\newcommand{\ua}{\mbox{u.\,a.\ }}
\newcommand{\zB}{\mbox{z.\,B.\ }}
\newcommand{\dahe}{\mbox{d.\,h.\ }}
\newcommand{\Vgl}{Vgl.\ }
\newcommand{\bzw}{bzw.\ }
\newcommand{\evtl}{evtl.\ }
\newcommand{\ggf}{ggf.\ }

\newcommand{\Abbildung}[1]{Abbildung~\ref{fig:#1}}
\newcommand{\Tabelle}[1]{Tabelle~\ref{tab:#1}}
\newcommand{\Listing}[1]{Listing~\ref{lst:#1}}
\newcommand{\Kapitel}[1]{Kapitel~\ref{sec:#1}}
\newcommand{\Chapter}[1]{Kapitel~\ref{cha:#1}}

\newcommand{\bs}{$\backslash$}


% erzeugt ein Listenelement mit fetter Überschrift ------------------------------------------------------
\newcommand{\itemd}[2]{\item{\textbf{#1}}\\{#2}}


% einige Befehle zum Zitieren ---------------------------------------------------------------------------
\newcommand{\Zitat}[2][\empty]{%
	\ifthenelse{\equal{#1}{\empty}}%
	{\cite{#2}}%
	{\cite[#1]{#2}}%
}
%\newcommand{\Zitat}[2][\empty]{\ifthenelse{\equal{#1}{\empty}}{\citep{#2}}{\citep[#1]{#2}}}
\newcommand{\ZitatSpezialA}[3]{\citep[#1][#2]{#3}}
\newcommand{\ZitatSpezialB}[2]{\citep[#1][]{#2}}


% zum Ausgeben von Autoren
\newcommand{\AutorName}[1]{\textsc{#1}}
\newcommand{\Autor}[1]{\AutorName{\citeauthor{#1}}}


% verschiedene Befehle um Wörter semantisch auszuzeichnen -----------------------------------------------
\newcommand{\NeuerBegriff}[1]{\textbf{#1}}

%\newcommand{\Abkuerzung}[2]{\textit{#2}\nomenclature{#2}{#1}}

\NewDocumentCommand{\Abkuerzung}{m m o}{%
	\textit{#2}%
	\IfNoValueTF{#3}
	{\nomenclature{#2}{#1}}%
	{\nomenclature{#3}{#1}}%
}

\NewDocumentCommand{\AbkuerzungB}{m m o}{%
	#1\xspace%
	\textit{(#2)}%
	\IfNoValueTF{#3}
	{\nomenclature{#2}{#1}}%
	{\nomenclature{#3}{#1}}%
}

%\newcommand{\Fachbegriff}[2][\empty]{%
%	\ifthenelse{\equal{#1}{\empty}}%
%	{\textit{#2}\xspace}%
%	{\textit{#2}\xspace\footnote{#1}\nomenclature{#2}{#1}}%
%}

\NewDocumentCommand{\Fachbegriff}{m m o}{%
	\textit{#2}\xspace%
	\footnote{#1}%
	\IfNoValueTF{#3}
	{\nomenclature{#2}{#1}}%
	{\nomenclature{#3}{#1}}%
}

\NewDocumentCommand{\FachbegriffT}{m m o}{%
	\textit{#2}\xspace%
	\tablefootnote{#1}%
	\IfNoValueTF{#3}
	{\nomenclature{#2}{#1}}%
	{\nomenclature{#3}{#1}}%
}


\newcommand{\FachbegriffSpezialA}[4]{\textit{#4}\footnote{#3}\label{fn:#1}\nomenclature{#4}{#2. Siehe auch Fu{\ss}zeile auf Seite~\pageref{fn:#1}.}}

\newcommand{\FachbegriffSpezialB}[5]{\textit{#5}\footnote{#3}\label{fn:#1}\nomenclature{#4}{#2. Siehe auch Fu{\ss}zeile auf Seite~\pageref{fn:#1}.}}


% Beträge mit Währung -----------------------------------------------------------------------------------
\newcommand{\Betrag}[2][general]{#2\,\ifthenelse{\equal{#1}{dollar}}{\$}{}\ifthenelse{\equal{#1}{euro}}{€}{}\ifthenelse{\equal{#1}{yen}}{¥}{}\ifthenelse{\equal{#1}{cent}}{¢}{}\ifthenelse{\equal{#1}{pound}}{£}{}\ifthenelse{\equal{#1}{peso}}{₱}{}\ifthenelse{\equal{#1}{baht}}{฿}{}\ifthenelse{\equal{#1}{franc}}{₣}{}\ifthenelse{\equal{#1}{lira}}{₤}{}\ifthenelse{\equal{#1}{drachma}}{₯}{}\ifthenelse{\equal{#1}{pfennig}}{₰}{}\ifthenelse{\equal{#1}{general}}{¤}{}}


% Sonstiges ---------------------------------------------------------------------------------------------
\newcommand{\Eingabe}[1]{\texttt{#1}}
\newcommand{\Code}[1]{\texttt{#1}}	% Monotype oder Type Writer
\newcommand{\Datei}[1]{\texttt{#1}}

\newcommand{\Datentyp}[1]{\textsf{#1}}
\newcommand{\XMLElement}[1]{\textsf{#1}}
\newcommand{\Webservice}[1]{\textsf{#1}}


% Beschriftung von Tabellen und Bildern ändern ----------------------------------------------------------
\addto\captionsngerman{
	\renewcommand{\figurename}{Abb.}
	\renewcommand{\tablename}{Tab.}
}

% Hinweis auf Unterstütztes schreiben durch AI
\newcommand{\AI}{\footnote{Die sprachliche \"Uberarbeitung dieses Kapitels erfolgte unter Zuhilfenahme von KI-Sprachmodellen. Inhaltliche Aussagen und Schlussfolgerungen stammen ausschlie{\ss}lich vom Autor.}}


% Spaltendefinition rechtsbündig mit definierter Breite -------------------------------------------------
\newcolumntype{w}[1]{>{\raggedleft\hspace{0pt}}p{#1}}


% Linksbündige Tabellenspalten mit tabularx -------------------------------------------------------------
\newcolumntype{y}[1]{>{\RaggedRight\arraybackslash\hsize=#1\hsize}X}
